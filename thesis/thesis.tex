% This work is licensed under the Creative Commons Attribution-ShareAlike 4.0 International License.
% To view a copy of this license, visit http://creativecommons.org/licenses/by-sa/4.0/.

\documentclass[a4paper,12pt]{scrbook}
\usepackage[T1]{fontenc} % Include italic fonts
\usepackage{fontspec} % Compile with \XeLaTeX
\usepackage{geometry} % Page margins
\usepackage{titling} % Title page container
\usepackage{wrapfig} % Picture container
\usepackage{graphicx} % Including graphics
\usepackage{natbib}
\usepackage[colorlinks, allcolors=blue,xetex]{hyperref}
\usepackage{hyperxmp} % Import license metadata
\usepackage[acronym,toc,shortcuts,nohypertypes=acronym]{glossaries} % Acronyms
\renewcommand{\familydefault}{\sfdefault}
\bibliographystyle{apalike}

\title{Evaluating the potential of Sentinel-2 and Landsat image time series for detecting selective logging in the Amazon}
\author{Dainius Masiliūnas}
\date{\today}
\hypersetup{
    pdflicenseurl={http://creativecommons.org/licenses/by-sa/4.0/},
    pdfcopyright={This work is licensed under the Creative Commons Attribution-ShareAlike 4.0 International License.},
    pdfauthor={\theauthor}, % These are supposed to be the default but don't seem to be
    pdftitle={\thetitle},
    pdflang={en-GB}
}

% Define acronyms
\newcommand{\underletter}[1]{\textbf{#1}} % Quick toggle of underlines
\newacronym{NDVI}{NDVI}{\underletter{N}ormalised \underletter{D}ifference \underletter{V}egetation \underletter{I}ndex}
\newacronym{REDD+}{REDD+}{\underletter{R}educing \underletter{E}missions from \underletter{D}eforestation and forest \underletter{D}egradation}
\newacronym{LAI}{LAI}{\underletter{L}eaf \underletter{A}rea \underletter{I}ndex}
\newacronym{EVI}{EVI}{\underletter{E}nhaced \underletter{V}egetation \underletter{I}ndex}
\newacronym{OLI}{OLI}{\underletter{O}perational \underletter{L}and \underletter{I}mager}
\newacronym{NBR}{NBR}{\underletter{N}ormalised \underletter{B}urn \underletter{R}atio}
\newacronym{NBR2}{NBR2}{\underletter{N}ormalised \underletter{B}urn \underletter{R}atio \underletter{2}}
\newacronym{NDMI}{NDMI}{\underletter{N}ormalised \underletter{D}ifference \underletter{M}oisture \underletter{I}ndex}
\newacronym{SAVI}{SAVI}{\underletter{S}oil \underletter{A}djusted \underletter{V}egetation \underletter{I}ndex}
\newacronym{MSAVI}{MSAVI}{\underletter{M}odified \underletter{S}oil \underletter{A}djusted \underletter{V}egetation \underletter{I}ndex}
\newacronym{TM}{TM}{\underletter{T}hematic \underletter{M}apper}
\newacronym{ETM+}{ETM+}{\underletter{E}nhanced \underletter{T}hematic \underletter{M}apper \underletter{Plus}}
\newacronym{WUR}{WUR}{\underletter{W}ageningen \underletter{U}niversity \& \underletter{R}esearch}
\newacronym{GPS}{GPS}{\underletter{G}lobal \underletter{P}ositioning \underletter{S}ystem}
\newacronym{GNSS}{GNSS}{\underletter{G}lobal \underletter{N}avigation \underletter{S}atellite \underletter{S}ystem}
\newacronym{NIR}{NIR}{\underletter{N}ear \underletter{I}nfra\underletter{r}ed}
\newacronym{SWIR}{SWIR}{\underletter{S}hort\underletter{w}ave \underletter{I}nfra\underletter{r}ed}
\newacronym{RIL}{RIL}{\underletter{R}educed-\underletter{I}mpact \underletter{L}ogging}
\newacronym{USGS}{USGS}{\underletter{U}.\underletter{S}. \underletter{G}eological \underletter{S}urvey}
\newacronym{UTM}{UTM}{\underletter{U}niversal \underletter{T}ransverse \underletter{M}ercator}
\newacronym{ESA}{ESA}{\underletter{E}uropean \underletter{S}pace \underletter{A}gency}
\newacronym{MSI}{MSI}{\underletter{M}ulti-\underletter{s}pectral \underletter{I}mager}
\newacronym{ASTER}{ASTER}{\underletter{A}dvanced \underletter{S}paceborne \underletter{T}hermal \underletter{E}mission and Reflection \underletter{R}adiometer}
\newacronym{SLC}{SLC}{\underletter{S}can \underletter{L}ine \underletter{C}orrector}
\newacronym{LP DAAC}{LP DAAC}{\underletter{L}and \underletter{P}rocesses \underletter{D}istributed \underletter{A}ctive \underletter{A}rchive \underletter{C}enter}
\newacronym{ESPA}{ESPA}{Earth Resources Observation and Science (\underletter{E}ROS) Center \underletter{S}cience \underletter{P}rocessing \underletter{A}rchitecture}
\newacronym{RGB}{RGB}{\underletter{R}ed, \underletter{G}reen, \underletter{B}lue}
\newacronym{DBH}{DBH}{\underletter{D}iameter at \underletter{B}reast \underletter{H}eight}
\makenoidxglossaries

\geometry{top=1.25cm,bottom=0.96cm,inner=2cm,outer=1.91cm,foot=0cm,includeheadfoot} % First page margins
\begin{document}
 \begin{titlingpage}
  {\Large Geo-information Science and Remote Sensing}\vspace{0.9cm}
  
  {\Large Thesis Report GIRS-2017-xx}\vspace{0.9cm}
  
  \hrule\vspace{1.1cm}
  
  {\bfseries \Large \MakeUppercase{\thetitle}}\vspace{2.0cm}
  
  \begin{wrapfigure}{r}{0.55\textwidth}
    \vspace{1cm}
    \includegraphics[height=9.5cm,draft]{title/picture.png}
  \end{wrapfigure}
  
  {\Large \theauthor}\vspace{5.5cm}
  
  \rotatebox{90}{\Large \thedate}\vspace{1.5cm}
  
  \sloppypar{\hspace{-2cm}\includegraphics[width=13cm]{title/WUR_RGB_standard}}
  
  \sloppypar{\noindent\makebox[\textwidth]{\hspace*{\dimexpr\evensidemargin-\oddsidemargin}\includegraphics[width=\paperwidth]{title/image2}}}
  
  \newgeometry{top=1.25cm,bottom=1.25cm,inner=1.91cm,outer=1.91cm,foot=1.19cm,includeheadfoot} % Subsequent page margins
  \thispagestyle{empty}
  
  \begin{center}
  {\bfseries \Large \thetitle}\vspace{2.7cm}
  
  {\Large \theauthor}\vspace{1.1cm}
  
  {Registration number 93 04 07 546 120}\vspace{3.5cm}
  
  {\large \underline{Supervisors}:}\vspace{1.1cm}
  
  {Dr Jan Verbesselt}
  
  {Dr Marielos Peña-Claros}\vspace{3.0cm}
  
  {A thesis submitted in partial fulfilment of the degree of Master of Science}
  
  {at Wageningen University and Research Centre,}
  
  {The Netherlands.}\vspace{2.7cm}
  \end{center}
  
  \begin{flushright}
    {\thedate}
  
    {Wageningen, The Netherlands}
  \end{flushright}\vspace{0.5cm}

    Thesis code number: GRS-80424
  
    Thesis Report: GIRS-2017-xx
  
    {Wageningen University and Research Centre}
  
    {Laboratory of Geo-Information Science and Remote Sensing}
 \end{titlingpage}

\newgeometry{top=1.25cm,bottom=1.25cm,inner=1.91cm,outer=1.91cm,foot=1.19cm,includeheadfoot}
\chapter*{Abstract}

\textbf{Keywords:} 

\tableofcontents

\chapter{Introduction}

% What is selective logging
Selective logging is a process by which trees in a forest are cut down according to specific criteria rather than indiscriminately. This type of logging has become common practice in tropical forest areas, where high biodiversity leads to a highly complex forest structure with trees of different sizes, shapes and properties. Trees that are commercially viable for logging are typically of a particular species or size, and tend to be spread over an area, mixed with trees of lower commercial value. Once a large tree is cut down, it is likely to fall on neighbouring trees. This causes a disturbance in the area, especially if only the tree stem is extracted, leaving the canopy (large branches and foliage) on top of the understory and forest floor.

% Why is it important to monitor it
Selective logging contributes to forest degradation, as per the United Nations \ac{REDD+} programme. Selectively logged forests most often remain forests after the logging event \citep{asner_condition_2006}, but they are disturbed, as wood from large trees is removed from the ecosystem. Succession is initiated at the points of disturbance, resulting in the replacement of mature, climax community trees with young pioneer species. Selective logging may be done illegally \citep{rutishauser_rapid_2015}, in which case the result is loss of biodiversity and lowered forest resilience. However, it could also be part of sustainable forestry practices, in which case trees are selected and logged in a way that reduces this negative impact to the forest and creates favourable conditions for succession in managed forests \citep{west_forest_2014, keller_4._2004}. This practice is called \ac{RIL}. Selective tree dieoff may also be caused by natural events, such as windstorms and landslides \citep{frolking_forest_2009}. In all cases, monitoring such forest degradation is a key part of the \ac{REDD+} programme. It is important to monitor such activity in order to minimise illegal logging and allow for more precise estimation of existing and historical carbon stocks \citep{piponiot_carbon_2016, pinard_simulated_2000}.

% Why treefall gaps
Even though selective logging is widespread in tropical forests, little is known about the real scale of logging activities. Selective logging is difficult to detect and quantify due to the remoteness of the affected forests, problematic accessibility, and limited traces of selective logging activities, which disappear over time due to regrowth. While remote sensing approaches have been employed for detecting selective logging in the past \citep{shimizu_using_2017, frolking_forest_2009, broadbent_recovery_2006, keller_4._2004}, their success has been limited. A typical modern selective logging operation leaves four types of traces that are potentially detectable using remote sensing: logging roads, which are built for easier extraction of logs by trucks; log decks, which are clearings in which logs are stored until they can be picked up by trucks; skid trails, which are made by skidders while attempting to gain access to the selective logging site and dragging the logs to the log decks; and treefall gaps, which are formed in the forest canopy when a tree is cut down \citep{asner_remote_2002}. The two former logging features are rather distinct and have been successfully detected from satellite imagery, however, their prevalence is not necessarily a direct indication of the magnitude of the performed selective logging activities. In some selective logging campaigns, only minor roads are made and log decks are omitted entirely; skidders can carry logs one by one to the nearest existing infrastructure, rather than mandating the creation of new roads and log decks \citep{read_spatial_2003}. Thus treefall gaps and skid trails are the most direct evidence of selective logging. In comparison to all of the other logging features, treefall gaps are by far the most spatially extensive \citep{asner_remote_2002}.

The detection of treefall gaps and skid trails from optical satellite imagery so far has been challenging. While there are some cases in which the treefall gaps have reportedly been detectable \citep{frolking_forest_2009}, either very high resolution, long revisit time sensors were needed \citep{read_spatial_2003}, or the differences in reflectance compared to unlogged forest were found to be minimal \citep{asner_canopy_2004, broadbent_recovery_2006}. With the advent of new satellites and sensors, such as Landsat 8 and Sentinel-2, the ability to detect such small variations in the canopy should improve due to higher spatial resolutions and more frequent revisit times. In the tropical regions, the revisit time is particularly important due to high cloud cover during the rainy season, which inhibits monitoring selective logging sites shortly after the logging event has happened.

% What we could do if we did detect
If it is doable to detect and track individual treefall gaps in time, it would also allow for estimating the length of succession in the particular area. While full secondary forest regrowth after logging may take up to 125 years \citep{rutishauser_tree_2016}, the canopy closure at the selective logging site happens much faster, estimated at 6-3 months depending on the size of the disturbance \citep{broadbent_recovery_2006}. Precise knowledge of the recovery time after selective logging is important directly for the countries in the area for setting logging policies, as well as indirectly for climate change modelling \citep{rutishauser_rapid_2015}. However, so far such estimation has been challenging \citep{piponiot_carbon_2016} and thus it has only been performed based on chronosequences (forest plots of different timespan since logging) \citep{broadbent_recovery_2006} or extrapolated from recovery rates \citep{rutishauser_rapid_2015}. Making use of the full archive of satellite imagery to construct a time series would allow for a more precise estimate of regrowth time as well as per-tree statistics, while eliminating plot-specific effects. In the long run, a system could be developed that detects such selective logging events and provides for both near-real-time monitoring of selective logging events and information on whether and when selective logging has occurred in the past. This would in turn enhance the knowledge on the current and past state of tropical forests, provide information on how widespread selective logging is and help quantify its effects on the tropical forest ecosystem. In addition, knowing the location of disturbances and their canopy closure times would allow for a more accurate large-scale estimation and monitoring of biomass and carbon stocks in tropical forests by enhancing tropical forest post-logging regrowth models \citep{herault_growth_2010}.

\chapter{Problem definition and research questions}

While satellite image time series trajectory analysis has been relatively well-established in the recent years, few studies focus on the detection of small-extent selective logging features (skid trails and treefall gaps) using this technique. Most studies of tropical forest regrowth make use of chronosequences of plots with different age instead. However, time series analysis is well-suited for this purpose, because it allows detecting the disturbance time and regeneration length in a much more precise manner. In addition, time series analysis is not affected by site-specific effects, as is the case with chronosequences.

Another advantage of using satellite imagery for detecting selective logging is that it is spatially exhaustive. Satellites constantly monitor surface reflectance over the entire globe, it is not limited to a select number of test plots. With the advent of the Sentinel-2 programme, the public now has access to data that is of much finer spatial resolution ($10\times10$ m, as opposed to $30\times30$ m of Landsat) while maintaining a short revisit time. This new data may allow better detection and monitoring of selective logging sites, including treefall gaps and skid trails, compared to what was possible before.

The goal of this thesis is to evaluate the potential and added value of new satellite imagery for detecting selective logging events by employing time series detection methods on Sentinel-2 and Landsat 8 imagery in select areas in the Amazon region. The research questions that the thesis aims to answer are:

\begin{enumerate}
 \item How well can selective logging events be detected by employing time series methods on optical satellite imagery?
 \item What is the sensitivity of different vegetation indices for detecting selective logging treefall gaps?
\end{enumerate}

\chapter{Data and methods}

% TODO: Check whether it's worth mentioning Bolivia and Planet Labs; what is the name of the Planet Labs satellite; more references?
In order to assess the detectability of selective logging events, three areas of interest were selected in the Amazon rainforest where selective logging is known to have taken place. Five vegetation indices derived from Landsat 7, Landsat 8 and Sentinel-2 imagery were used to construct time series in order to test detection of selective logging features, especially treefall gaps. High resolution imagery from the \ac{ASTER} programme and very high resolution imagery from Planet Labs PlanetScope satellite constellation and DigitalGlobe satellites were used for validation.

\section{Sensor characteristics}

\subsection{Landsat 7, Landsat 8}

Data from two Landsat missions was used in order to obtain long-term dense time series of ground observations at 30 m resolution. Landsat 7, launched in 1999, carries the \ac{ETM+} instrument that captures imagery in 3 visible light bands, one \ac{NIR} band, two \ac{SWIR} bands, one thermal band and one broadband panchromatic band \citep{u.s._geological_survey_product_2017_2}. Landsat 8, launched in 2013, carries the \ac{OLI} instrument that captures imagery in comparable bands to \ac{ETM+}, but also adds an additional thermal band as well as coastal aerosol and cirrus bands \citep{u.s._geological_survey_product_2017}. Both of the satellites have a revisit time (at the study sites) of around two weeks, and continue transmitting images to this day, although in 2003 Landsat 7 experienced an \ac{SLC} failure that resulted in it transmitting 22\% less data than before. The Landsat collection data is available from the \ac{USGS} \ac{ESPA} system preprocessed into ground reflectance (Level 2A), as well as vegetation indices derived from the reflectance products. The products are in the \ac{UTM} coordinate system.

\subsection{Sentinel-2}

Sentinel-2 is a satellite launched in 2016 by the \ac{ESA} that carries the \ac{MSI} instrument which captures 12 optical bands at varying spatial resolutions: the three visible bands and the \ac{NIR} band are available in 10 m resolution, the two \ac{SWIR}, three red edge bands and narrow-band \ac{NIR} in 20 m resolution, and coastal aerosol, cirrus and water vapour bands at 60 m resolution \citep{suhet_sentinel-2_2015}. The Sentinel-2 products are available from the \ac{ESA} Sentinel Data Hub at the Level 1C processing level (top-of-atmosphere radiance).

\subsection{ASTER, DigitalGlobe}

\ac{ASTER} is a joint NASA-Japan government mission and an instrument (comprised of three separate sensors) on board the Terra satellite, launched in 1999, which captures images in 14 spectral bands with varying resolutions: 15 m for the two visible bands (green-yellow and red) and two \ac{NIR} bands (nadir and backwards-facing), 30 m for six \ac{SWIR} bands and 90 m for five thermal bands. The revisit time for the images taken during the day varies from 11 to 0 times per year. Since 2016, the data is freely available through the \ac{USGS} \ac{LP DAAC} as Level 2 (surface reflectance) products \citep{nasa_lp_daac_aster_2006}. On April 6, 2008 the \ac{SWIR} sensor overheated, resulting in completely saturated and thus unusable \ac{SWIR} bands from that date up to the present \citep{meyer_advanced_2015}.

Images from the DigitalGlobe satellite fleet (GeoEye-1, WorldView-2), available through Google Earth, were also used for validation. These images are displayed in \ac{RGB} and have a pan-sharpened spatial resolution of 0.5 m. The revisit time in the areas of interest vary but is very long (several years per visit).

\section{Vegetation indices}

Five vegetation indices, as defined by \citet{u.s._geological_survey_product_2017_3}, were compared in this study. \ac{SAVI} and \ac{NBR2} were not used due to their similarity with \ac{MSAVI} and \ac{NBR}, respectively. In the case of Landsat imagery, precomputed vegetation indices were used, and for the other imagery, vegetation indices were derived from the imagery manually.

\subsection{NDVI}

\ac{NDVI} is a commonly used vegetation index that is a ratio between the red and \ac{NIR} bands:

$$ NDVI = \frac{\rho_{NIR} - \rho_{Red}}{\rho_{NIR} + \rho_{Red}} $$

where $\rho_{NIR}$ is the surface reflectance in the spectral band centred around 830 nm and $\rho_{Red}$ is the surface reflectance in the spectral band centred around 660 nm \citep{tucker_monitoring_1979}.

\ac{NDVI} is easy to interpret and has a range of 0-1, but it is known to be insensitive to small changes in areas with dense vegetation, as the vegetation index saturates \citep{huete_modis_1999}. In addition, cloud shadows over vegetation have \ac{NDVI} values close to 1, because the values of the red band are close or equal to zero.

\subsection{EVI, MSAVI}

In order to overcome the shortcomings of \ac{NDVI}, two more complex vegetation indices have been developed: \ac{EVI} and the \ac{SAVI} family. \ac{EVI} is designed to not saturate with high biomass the way NDVI does, and reduce atmospheric and background noise (for instance, thin clouds are compensated for). It requires the use of the blue visible band and several coefficients, and in this study the default ones adopted by NASA were used:

$$ EVI = 2.5 \cdot \frac{\rho_{NIR} - \rho_{Red}}{\rho_{NIR}+6 \cdot \rho_{Red} - 7.5 \cdot \rho_{Blue} + 1} $$

Here, $\rho_{Blue}$ is the surface reflectance in the blue band (centred around 485 nm in \ac{ETM+}) \citep{huete_modis_1999}. Cloud shadows have a low EVI value, as opposed to NDVI.

The \ac{SAVI} family was developed for solving the issue of vegetation on different soil backgrounds having an effect on NDVI. \ac{MSAVI} is a modification of \ac{SAVI} to maximise the reduction in soil background effects and increase the dynamic range of the vegetation signal \citep{qi_modified_1994}. In this case the coeffiecients employed by NASA were used as well:

$$ MSAVI = \frac{2 \cdot \rho_{NIR} + 1 - \sqrt{(2 \cdot \rho_{NIR} + 1)^2 - 8 \cdot (\rho_{NIR} - \rho_{Red})}}{2} $$

In the study area, \ac{MSAVI} visually appears very similar to \ac{EVI}, the main apparent difference is the lack of atmospheric noise reduction in \ac{MSAVI}. However, \ac{MSAVI} does not require the blue band, which allows calculation of \ac{MSAVI} with the \ac{ASTER} sensor that does not capture the blue band.

\subsection{NDMI, NBR}

\ac{NDMI} and \ac{NBR} are vegetation indices that are a normalised ratio between reflectance in the blue band and the \ac{SWIR} band \citep{key_normalized_2002}. The difference is the wavelength of the \ac{SWIR} band:

$$ NDMI = \frac{\rho_{NIR} - \rho_{1600}}{\rho_{NIR} + \rho_{1600}} $$

$$ NBR = \frac{\rho_{NIR} - \rho_{2220}}{\rho_{NIR} + \rho_{2220}} $$

$\rho_{1600}$ indicates surface reflectance in the \ac{SWIR} band centered around 1600 nm, whereas $\rho_{2220}$ indicates surface reflectance in the \ac{SWIR} band centered around 2220 nm. These vegetation indices are related to moisture in that water interferes with reflectance in the \ac{SWIR} region, and dry leaf biomass has a similar reflectance in \ac{NIR} as in \ac{SWIR}, whereas a healthy leaf has lower reflectance in \ac{SWIR} \citep{cibula_response_1992}.

\ac{NBR} is commonly used for selective logging detection, as it is perceived to be more sensitive to disturbance events \citep{schneibel_assessment_2017, shimizu_using_2017}. Both \ac{NBR} and \ac{NDMI} are insensitive to cloud shadows, since they appear to lower the reflectance in both \ac{NIR} and \ac{SWIR} equally.

\section{Field campaigns}

In order to be able to detect treefall gaps and other selective logging features, the exact location of trees that have been selectively logged and the time of logging was needed. For this purpose, the metadata of selective logging site lidar scans by \citet{gonzalez_de_tanago_estimation_2017} was used. In three areas during different years, selective logging campaigns were organised by the governments of the respective countries, and a team of researchers from \ac{WUR} went to take lidar scans of a number of trees that were selected to be cut down. Not all selectively logged trees were scanned, and a number of scans were taken of trees that did not get logged. In this study, only the metadata (site location information) of the lidar scans were used, not the lidar scan data itself.

\subsection{Peru, 2013}

\begin{figure}
    \centering
    \includegraphics[width=0.5\textwidth]{thesis-figures/01-peru-sites}
    \caption{The location of known selective logging sites and canopy gaps in the Peru study site, overlaid on Google satellite imagery.}
    \label{fig-peru}
\end{figure}

The first field campaign was in November of 2013 in the Madre de Dios region of Peru. The selective logging operation took place in a forest east of the village Planchón, which is 37 km north of the region's capital city Puerto Maldonado (see figure \ref{fig-peru}). This region has a dry and a wet season, the latter spans from December to March. In the area, shifting cultivation with cattle ranging and papaya plantations is encroaching upon previously intact forest, due to the proximity to the Interoceanic Road that crosses Puerto Maldonado and goes through Planchón to Bolivia and Brazil. The selective logging campaign was carried out by small farmland owners who obtained a grant for selective logging in the area, but did not have logging equipment such as skidders (logs were transported on motorcycles). The larger trees in the study area have a canopy diameter of 20-30 m.

Nine trees to be cut down were imaged from 13 positions arranged in a rectangle whose centre is the approximate location of where the tree crown would fall after logging. The location of each of the positions was determined using two separate Garmin \ac{GPS} devices. After logging, the same sites were imaged a second time. In this study, the centroid of all the measured points for each imaging site was used as a the location of the trees that were logged. Both dates before and after logging were recorded. The actual logging event happened in between the two dates, which were between 2 and 6 days apart.

In addition to the logged tree data, there were also point locations of canopy gaps (seven in total) that are likely to have been the result of previously logged trees in the selective logging campaign (but may also have been created in earlier selective logging campaigns). In the case of gaps, only the post-logging date was recorded. In addition to \ac{GPS} data, there was also data on the nine main sites in a database, however, the locations did not match the locations measured by GPS. The entries in the database were offset from the GPS measurements by 540 metres to the north. Therefore in total 25 points from the Peru campaign were analysed for changes in reflectance following a selective logging event.

\subsection{Guyana, 2014}

The second field campaign took place in the east of the Cuyuni-Mazaruni region of Guyana in November 2014. The nine selectively logged trees that were imaged are located in the Wineperu concession, 40 kilometres south of the region's capital, Bartica. The town is connected to the selective logging sites by a road. The sites are very close to the road, 300 m away at most. Like the Peru campaign, the trees were imaged before and after the selective logging event.

% TODO: How large are the trees? Selective logging lidar info?

\subsection{Guyana, 2017}

The third field campaign took place in the East Berbice-Corentyne region of Guyana in January 2017. The area where the trees were imaged is a very remote area at the end of a logging access road originally built in 1997 by the UNAMCO logging company, going south from the town of Kwakwani. After UNAMCO was closed in 2007, the road went unused until 2013, when a new logging company bought the concession and extended the road. For ease of timber extraction, a number of new primary logging roads (20 m across, gravel) were built in the logging site starting from 2014. Kwakwani is 85 kilometres north of the selective logging site.

In this campaign, a large number of inventoried trees were imaged, but only 16 of them were later selectively logged. In addition, no post-logging scans were made. Furthermore, in this campaign, a number of trees with different heights and \ac{DBH} were inventoried at the same points. Some of the trees to be cut down were small and below the canopy of larger trees.

% TODO: Were the trees also all cut down at once, or only a few of the inventoried ones get cut down? Could explain the large gaps with soil. Any data on the height?

\section{Preprocessing workflow}

\subsection{Top-of-canopy reflectance (Level 2A) processing}

\subsection{Cloud masking}

\subsection{Mosaicking}

\subsection{Time series stacking}

\section{Time series analysis}

\chapter{Results}

\section{Gap detectability}

Automation possibilities, pictures of areas, some telling time series examples, Landsat-to-Sentinel comparison

\section{Vegetation index comparison}

Sensitivity, threshold values, correlation, images at small scale and large scale

\chapter{Discussion}

\section{Selective logging detection possibilities}

What is possible to detect, in a case study format; this should answer the first research question

\subsection{Moderate size gaps in dense forests with a soil background}

Case of Guyana 2017: cutting down a tree makes soil appear, visible in \ac{NDVI}

\subsection{Small gaps in dense forests}

Case of Guyana 2017: cutting down a tree results in a shift in shadows, visible in \ac{NDMI}

\subsection{Clear-cuts}

Case of Peru 2013: clear-cuts highly affect \ac{NDVI} for a year, followed by an increase in \ac{NDVI} compared to the beginning due to less shadowing

\subsection{Logging roads}

Case of Guyana 2017: logging roads are easy to detect, but not skid trails. In other cases logging roads are not detected either (only skid trails used/under canopy/too narrow)

\section{Comparison of vegetation indices}

Which VI is suited for a particular case

\section{Comparison of sensors}

Analysis of Landsat vs ASTER vs Sentinel-2; which one to use for a particular use-case

\section{Comparison with other studies}

Why did others succeed (reproducibility, validation importance)

\section{Treefall gap detection challenges}

\subsection{Tree seasonality}

``White'' trees, fluctuation in VIs through the year

\subsection{Sparse forests}

Forests with enough light in the understory, or single trees, coupled with \ac{RIL}: no change or even increase in VIs

\subsection{Existing forest gaps}

When a tree is cut down next to an existing clearing/road/logging gap, especially in \ac{RIL}

\subsection{Understory tree logging}

When trees are below the canopy of other trees (Alvaro mentioned this in Guyana 2017)

\subsection{Remote sensing imagery spatial resolution}

Sentinel-2 vs Landsat vs Ikonos et al.

\subsection{Cloud cover}

Especially problematic in the Amazon, case of Peru 2013

\subsection{Time series length}

Case of Sentinel-2: not enough to build a stable history yet

\subsection{Data volume}

Big data, how much space and time it takes to process it all

\section{Recommendations}

What would be possible to improve so as to make detection feasible in the future; perhaps summarise this in a table with listed pros and cons

\subsection{Validation data}

In order to draw better conclusions, more data from larger selective logging campaigns would be needed; also the availibility of more high-resolution sensors helps (like ASTER)

\subsection{Cloud masking}

Perform better cloud and shadow detection (classification/temporal or combination of those)

\subsection{Spatial features}

Other vegetation indices to try; texture analysis; specifics of the extent of a single tree; using the features in time series

\subsection{Post-classification/$F_{Cover}$ comparisons}

Using machine learning for supervised (fuzzy) classification and then compare the results between different times; $F_{Cover}$ is proposed by Copernicus.

\subsection{Higher spatial resolution data}

Ikonos, QuickBird, GeoEye, etc.; using object-oriented methods

\subsection{Multi-sensor data}

Radar and lidar would not have problems with cloud cover, possible sensor fusion (also for classification)

\chapter{Conclusion}

\begin{enumerate}
 \item Possible to detect large tree gaps using Sentinel-2, almost impossible with Landsat; automation will be possible in the future as Sentinel-2 time series grow longer
 \item \ac{NDMI} is most sensitive to shadows, so useful for detecting small gaps, but at a risk of false positives; \ac{NDVI} is sensitive to bare soils, so more reliable, but needs higher resolution and a dense forest
\end{enumerate}

\chapter{Acknowledgements}

Jose and Alvaro for the data

\printnoidxglossary[type=acronym]

\bibliography{bibliography}

\end{document}
